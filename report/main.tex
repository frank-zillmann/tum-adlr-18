\documentclass[conference]{IEEEtran}
\IEEEoverridecommandlockouts
% The preceding line is only needed to identify funding in the first footnote. If that is unneeded, please comment it out.
%Template version as of 6/27/2024

\usepackage{cite}
\usepackage{amsmath,amssymb,amsfonts}
\usepackage{algorithmic}
\usepackage{graphicx}
\usepackage{textcomp}
\usepackage{xcolor}
\def\BibTeX{{\rm B\kern-.05em{\sc i\kern-.025em b}\kern-.08em
    T\kern-.1667em\lower.7ex\hbox{E}\kern-.125emX}}

% Added manually:
\usepackage{tikz}
\usetikzlibrary{arrows.meta, positioning}
\usepackage{hyperref}
    
\begin{document}

\title{ADLR-Project 18: Efficient Environment Exploration and 3D Reconstruction with Reinforcement Learning and Multiple View Geometry}

% 1\textsuperscript{st} % could be used for 1.
\author{\IEEEauthorblockN{Frank Zillmann}
\IEEEauthorblockA{\textit{Department of Computer Engineering} \\
\textit{Technical University of Munich (TUM)}\\
Munich, Germany \\
frank.zillmann@tum.de}
}

\maketitle

\begin{abstract}
% This document is a model and instructions for \LaTeX.
% This and the IEEEtran.cls file define the components of your paper [title, text, heads, etc.]. *CRITICAL: Do Not Use Symbols, Special Characters, Footnotes, 
% or Math in Paper Title or Abstract.

The abstract.

\end{abstract}

% \begin{IEEEkeywords}
% component, formatting, style, styling, insert.
% \end{IEEEkeywords}

\section{Introduction}

\section{Methods}
\subsection{Robot Environment}

\subsection{Reconstruction Policy}

\subsection{Robot Policy}

\subsection{Reward Function}

\subsection{Training (Reinforcement Learning)}

\section{Results and Discussion}

\subsection{Iterative Improvement}

\subsection{Problems/Challanges}

\subsection{Final Performance}

\section{Conclusion}

\bibliographystyle{IEEEtran}
\bibliography{references}

\end{document}
